\documentclass{article}

% Pour utiliser toues les fonctions du clavier
\usepackage[utf8]{inputenc} % un package
\usepackage[T1]{fontenc}      % un second package

% Choix de la langue
\usepackage[francais]{babel}  % un troisième package
\setlength{\parindent}{0pt}

% Taille des marges
\usepackage[top=2.5cm, bottom=2cm, left=2.5cm, right=1.5cm]{geometry}

% Pour l'espace entre les lignes
\usepackage{setspace}
% Utilisation:
% Moyen:
% \begin{onehalfspace}
% \end{onehalfspace}
% Grand:
% \begin{doublespace}
% \end{doublespace}

% Changement des polices
\usepackage{charter}

% Pour afficher du code
\usepackage{verbatim}
\usepackage{moreverb}


\usepackage{titling}
\setlength{\droptitle}{-5em}   % This is your set screw

% Version 2
\usepackage{listings}

% Couleurs
\usepackage{color}
\usepackage[dvipsnames]{xcolor}
\usepackage{colortbl}

\title{%
	Laboratoires services \\
	\large Rapport laboratoire 6
}
\author{\bsc{Bulloni} Lucas \& \bsc{Wermeille} Bastien}
%\date{10 Novembre 2017}

% En-têtes et pieds de pages
\usepackage{fancyhdr}
 
\pagestyle{fancy}
\fancyhf{}
\rhead{\bsc{Bulloni} Lucas \& \bsc{Wermeille} Bastien}
\lhead{Réseau et application.bib}
\chead{Rapport Labo 2}
\cfoot{\thepage}

% Package pour la légende de la table
\usepackage{caption}

% Package de multi-colonnes
\usepackage{multicol}

% Package pour les images
\usepackage{graphicx}

%bibliographie
\usepackage{csquotes}


% Pour les listes
\usepackage{enumitem}
\setlist[itemize]{topsep=0pt,after=\newline}

\definecolor{dkgreen}{rgb}{0,0.6,0}
\definecolor{gray}{rgb}{0.5,0.5,0.5}
\definecolor{mauve}{rgb}{0.58,0,0.82}

\lstset{frame=tb,
	language=bash,
	aboveskip=3mm,
	belowskip=3mm,
	showstringspaces=false,
	columns=flexible,
	basicstyle={\small\ttfamily},
	numbers=none,
	numberstyle=\tiny\color{gray},
	keywordstyle=\color{blue},
	commentstyle=\color{dkgreen},
	stringstyle=\color{mauve},
	breaklines=true,
	breakatwhitespace=true,
	tabsize=3
}

%affichage du titre au centre
\usepackage{titling}
\renewcommand\maketitlehooka{\null\mbox{}\vfill}
\renewcommand\maketitlehookd{\vfill\null}

%cache les liens moches
\usepackage[hidelinks]{hyperref}

\usepackage[backend=biber]{biblatex}
\addbibresource{biblio.bib}

%\bibliographystyle{plain}
\bibliography{biblio}

% Début du document
\begin{document}

\maketitle

\newpage

\tableofcontents

\newpage

\section{Introduction}
	
Dans le cadre d'un laboratoire du cours "Réseau et application", les services de base d'un réseau informatiques sont mis en pratiques. Les protocoles testés sont le DHCP, DNS, serveur web et optionnelement le protocole NTP. La partie NTP a été réalisée.

\subsection{Prérequis}

\begin{itemize}
	\item Un PC Linux avec NetKit
	\item Laboratoire netkit "qos"
\end{itemize}

\subsection{Réseau du laboratoire}

\subsubsection{Réseau initial}

Le réseau est composé de deux PC, dont un qui est un serveur web, ainsi qu'un serveur qui fait office de DNS et de DHCP.\\

Le nom du réseau est net1.mylan.ch.\\

[Image sous-réseau]

\subsubsection{Réseau final}

L'objectif final, est de dupliquer le premier réseau 3 fois et d'interconnecter les 4 sous-réseaux.\\

[Image réseau final]


\section{Déploiement d'un sous-réseau}

La première étape du laboratoire est la configuration du serveur DNS et DHCP. Toutes les manipulations de configurations ont été faites dans le fichier "n1-router.startup" afin que les modifications soit préservées lors du redémarrage du laboratoire.


\subsection{Plage d'adresses dynamiques}

La première étape est la configuration de la plage d'adresse IP dynamique. Toutes les machines configurées avec une adresse IP dynamique prendront une adresse entre 192.168.1.100/24 et 192.168.1.199/24.\\

L'adresse de broadcast est 192.168.1.255 et la passerelle est 192.168.1.1. Nous avons également ajouté le serveur DNS en prévoyance. Le serveur étant la même machine que le DHCP, l'adresse est également 192.168.1.1.\\

\begin{lstlisting}
subnet 192.168.1.0 netmask 255.255.255.0 {
	range 192.168.1.100 192.168.1.199;
	option routers 192.168.1.1;
	option broadcast-address 192.168.1.255;
	option domain-name-servers 192.168.1.1;
	option domain-name "net1.mylan.ch";
}
\end{lstlisting}


\subsubsection{Adresses statiques}


\section{Déploiement de 4 sous-réseaux}

\section{Autres services}

PTP text du cours:
Le protocole PTP (Precision Time Protocol ), coupe a du support dans les éuipements de commutation permettant de modifier les datagrammes couche 4 en transit (transparent clock) et d'y inserer des informations de délais efectifs, peut atteindre la précision requise, dans la mesure ou les délais peuvent être estimées symétriques et constants durant les fenêtres de synchronisation. Un des avantages du
PTP est la possibilitée, si les équipements de commutation le supportent (boundary clock), de pouvoir supporter la synchronisation en phase.

\section{Questions}

\section{Conclusion}

\printbibliography


\end{document}

